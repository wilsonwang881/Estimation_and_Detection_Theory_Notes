\documentclass[11pt,letterpaper,titlepage]{article}

\usepackage{geometry}
\geometry{left=1.5cm,right=1.5cm,top=1.5cm,bottom=2.5cm}

\usepackage{setspace}
\onehalfspacing

\usepackage{fancyhdr}

\usepackage{amsmath}

\usepackage{amssymb}

\usepackage{booktabs}

\usepackage{pifont}

\usepackage{hyperref}
\hypersetup{
    colorlinks,
    citecolor=black,
    filecolor=black,
    linkcolor=black,
    urlcolor=black
}

\usepackage{trace}

\pagestyle{fancy}
\lhead{}
\rhead{}
\lfoot{ECEN 662 Estimation and Detection Theory}
\cfoot{\thepage}
\rfoot{Notes @Lei Wang}
\renewcommand{\headrulewidth}{0pt}
\renewcommand{\headwidth}{\textwidth}
\renewcommand{\footrulewidth}{0.4pt}
\newcommand{\RomanNumeralCaps}[1]
    {\MakeUppercase{\romannumeral #1}}
    
\begin{document}

\begin{titlepage}
  \centering
	{\scshape\large Texas A\&M University \par}
	\vspace{1cm}
	{\scshape\Large Department of Electrical and Computer Engineering \par}
	\vspace{4cm}
    \vspace{0.5cm}
	{\huge\bfseries ECEN 662 Estimation and Detection Theory\par}
	\vspace{4cm}
	{\Large Notes\par}
	\vspace{1cm}
	{\Large Student: Lei Wang (Wilson)\par}
	\vspace{1cm}
	{\Large UIN: 829009485\par}
	\vspace{1cm}
	{\Large Instructor: Dr. Jean-Francois Chamberland\par}
	\vspace{4cm}
	\vfill

  % Bottom of the page
% 	{\large Submitted: February 11th, 2020 \par}

\end{titlepage}

\newpage

\tableofcontents{}

\newpage

\section{Lesson 1 1.14.2020}

Statistical inference.

\subsection{Probability Laws:}

\begin{enumerate}
    \item Non-negative
    
    \item Normalization: sum to 1
    
    \item Countable additivity: if disjoint, $Pr(\sum N)=\sum Pr(N)$
    
\end{enumerate}

\subsection{Sample Space:}

A sample space $\Omega$ contains the outcomes $\omega$, which are disjoint, mutually exclusive and collectively exhaustive.

A collection of subsets of $S$ is called a sigma algebra $B$ if it satisfies the 3 properties:

\begin{enumerate}
    \item $\phi \in B$
    
    \item If $A \in B$, then $A^C \in B$
    
    \item If $A_1, A_2... \in B$, then $\bigcup\limits_{i=1}^{\infty} \in B$ (closed under countable unions)
\end{enumerate}

From Wikipedia: In mathematical analysis and in probability theory, a $\sigma$-algebra (also $\sigma$-field) on a set $X$ is a collection $\sum$ of subsets of $X$ that includes $X$ itself, is closed under complement, and is closed under countable unions.

\newpage

\section{Lesson 2 1.16.2020}

\subsection{Conditional Probability:}

\begin{equation*}
    Pr(A|B) = \frac{Pr(A\cap B)}{Pr(B)} \in [0, 1] \text{, } Pr(A\cap B) \in [0, Pr(B)] \text{, } Pr(B) > 0
\end{equation*}

\begin{itemize}
    \item Restriction to the sample space
    
    \item $Pr(.|B)$: new probability laws, still need to satisfy the 3 axioms
    
    \item $Pr(B) = \sum Pr(B|A_i)Pr(A_i)$
\end{itemize}

\subsection{Countable Additivity:}

Disjoint $A_1$, $A_2$, $A_3$...

\begin{equation*}
    \begin{aligned}
        Pr(\bigcup\limits_{i=1}^{\infty} A_i) &= \sum_{i=1}^{\infty} Pr(A_i) \\
        Pr(\{\bigcup\limits_{i=1}^{\infty} A_i\}|B) &= \frac{Pr(\{\bigcup\limits_{i=1}^{\infty} A_i \}\cap B)}{Pr(B)} \text{, the numerator distributes the set over the union} \\
        &= \frac{Pr(\bigcup\limits_{i=1}^{\infty} \{A_i \cap B\})}{Pr(B)} \\
        &= \frac{\sum\limits_{i=1}^{\infty} Pr(A_i \cap B)}{Pr(B)} \\
        &= \sum\limits_{i=1}^{\infty} Pr(A_i | B) \\
    \end{aligned}
\end{equation*}

\subsection{Independence:}

\begin{itemize}
    \item $Pr(A \cap B) = Pr(A) Pr(B)$
    
    \item $Pr(A|B) = \frac{Pr(A \cap B)}{Pr(B)} = \frac{Pr(A)Pr(B)}{Pr(B)} = Pr(A)$, if independent and $Pr(B) > 0$
    
    \item $Pr(B|A) = Pr(B)$ if independent and $Pr(A) > 0$
\end{itemize}

\newpage

\section{Lesson 3 1.21.2020}

\subsection{Event Independence:}

\begin{itemize}
    \item $Pr(A_1 \cap A_2 \cap A_3) = Pr(A_1)Pr(A_2)Pr(A_3)$
    
    \item Also check each pair
    
    \item 3 sets, 4 conditions to check
    
    \item $k$ sets: $2^k - k -1$ conditions to check
    
    \item Why pairwise independence is not enough? i.e. throwing a coin twice
\end{itemize}

\subsection{Random Variable:}

\begin{itemize}
    \item $X: \Omega \rightarrow \mathbb{R}$, sample space to real numbers
    
    \item Induced probability law: $P_X(x \in A), A \subset \mathbb{R}$, $X^{-1} (A) \subset \Omega$:
    
    $P_X(x \in A) = Pr(x^{-1} (A))$, $x^{-1} (A)$ is the event
\end{itemize}

\subsection{CDF:}

\begin{itemize}
    \item $F_X (x) = Pr(X \leq x)$
    
    \item Discrete: PMF, steps, right-continuous
    
    \item Continuous: PDF, $F_X (x) = \int_{-\infty}^{x} f_X(t)dt$
    
    \item Mixed
\end{itemize}

\newpage

\section{Lesson 4 1.23.2020}

\subsection{Transformation:}

\begin{itemize}
    \item $Y = g(X) = g \circ X$: composition of functions
    
    $Y$ is a random variable
    
    $Pr(Y \in A)$
    
    $Y = g \circ X \rightarrow Pr(g(X) \in A) = Pr(x \in g^{-1}(A))$: $g^{-1}(A) \rightarrow$ event under $P_X$, the pre-image of $A$ under $g$
    
    \item Suupose $G$ is strictly increasing and continuous. How to you relate $f_Y(.)$ to $f_X(.)$?
    
    \begin{equation*}
        \begin{aligned}
            Pr(Y \leq y) &= Pr(x \in g^{-1}((-\infty, y])) \\
            &= Pr(x \in (-\infty, g^{-1} (y)]) \\
            &= F_X(g^{-1} (y))
        \end{aligned}
    \end{equation*}
\end{itemize}

\end{document}
