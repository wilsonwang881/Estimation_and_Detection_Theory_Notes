\documentclass[11pt,letterpaper,titlepage]{article}

\usepackage{geometry}
\geometry{left=1.5cm,right=1.5cm,top=1.5cm,bottom=2.5cm}

\usepackage{setspace}
\onehalfspacing

\usepackage{fancyhdr}

\usepackage{amsmath}

\usepackage{amssymb}

\usepackage{booktabs}

\usepackage{pifont}

\pagestyle{fancy}
\lhead{}
\rhead{}
\lfoot{ECEN 662 Estimation and Detection Theory}
\cfoot{\thepage}
\rfoot{Notes @Lei Wang}
\renewcommand{\headrulewidth}{0pt}
\renewcommand{\headwidth}{\textwidth}
\renewcommand{\footrulewidth}{0.4pt}
\newcommand{\RomanNumeralCaps}[1]
    {\MakeUppercase{\romannumeral #1}}
    
\begin{document}

\section{Class 1 1.14.2020}

Statistical inference.

\subsection{Probability Laws:}

\begin{enumerate}
    \item Non-negative
    
    \item Normalization: sum to 1
    
    \item Countable additivity: if disjoint, $Pr(\sum N)=\sum Pr(N)$
    
\end{enumerate}

\subsection{Sample Space:}

A collection of subsets of S is called a sigma algebra B if it satisfies the 3 properties:

\begin{enumerate}
    \item $\phi \in B$
    
    \item If $A \in B$, then $A^C \in B$
    
    \item If $A_1, A_2... \in B$, then $$
\end{enumerate}

\end{document}
